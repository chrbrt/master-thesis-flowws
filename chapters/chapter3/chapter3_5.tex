\section{Anforderungen}
Diese Sektion handelt über die funktionalen und nicht-funktionalen Anforderungen an flowws. Da es sich bei flowws um eine Konzeptarbeit und Machbarkeitstudia handelt, wird der Fokus auf nicht-funktionale Anforderungen, welche Bedingungen an Benutzbarkeit, Transparenz und an die in (kap) beschriebenen Cognitive Dimensions stellen. 

\subsection{Funktionale Anforderungen}\label{subsec:fanf}
\begin{table}[H]
\caption{Funktionale Anforderungen}
\label{tab:fanf}
\begin{tabularx}{\textwidth}{llX}
\hline
\rowcolor[HTML]{EFEFEF}
ID    & Name                       & Beschreibung \\ \hline
FA\#1 & Vergleichsoperationen      & Verechnung von reellwertigen Signalen durch ($<,\leq,=,\neq,\geq,>$) möglich. \\ \hline
FA\#2 & Bool'sche Operationen      & Verechnung von bool'schen Signalen durch ($\neg, \land, \lor, \oplus$) möglich. \\ \hline
FA\#3 & Zeitgestauerte Operationen & Zeitliche Manipulation von Signalen durch Verzögerung \\ \hline
FA\#4 & Auslesen von Sensordaten   & Kontinuierliches Auslesen von Sensordaten zur Weiterverarbeitung \\ \hline
FA\#5 & Ansteuern von Aktuatoren   & Ansteuern von Aktuatoren und Auslesen deren Zustände \\ \hline
FA\#6 & Konvertierung von Signalen & Konvertierung zwischen verschiedenen Wertebereichen (bspw. $\left [ 0;1,0 \right ] \rightarrow \left [ 10;37 \right ]$) \\ \hline
\end{tabularx}
\end{table}

\subsection{Nicht-funktionale Anforderung}

\subsubsection{Verständlichkeit}
\begin{table}[H]
\caption{NFN \#0}
\label{tab:nfn0}
\begin{tabularx}{\textwidth}{lX}
\hline
\rowcolor[HTML]{EFEFEF} 
NFN\#0        & Nichtfunktionale Anforderung \#0 \\ \hline
Name          & Paralleles und sequentielles Verhalten begreifbar machen \\ \hline
Beschreibung  & flowws ermöglicht dem Endnutzer ein Verständnis von parallelen aperiodischen Ereignissen und sequentiellen Vorgängen zu erlangen. \\ \hline
Rationale     & Parallele Programmierung ist selbst für Experten ein kognitiv anstrengender Prozess, allerdings unabdingbar für den asynchronen, aperiodischen Kontext von cBlocks. Gleichzeitig benötigt der Endnutzer ein solides Paradigma um sequentielle Steuerung von bspw. Aktuatoren zu garantieren. \\ \hline
Fit-Kriterium & Zur Erfüllung der Anforderung muss ein Endnutzer in wenigen Stunden das parallele und sequentielle Verhalten von fertigen Programmen gedeutet werden kann und später auch modifiziert werden kann.\\ \hline
\end{tabularx}
\end{table}


\begin{table}[H]
\caption{NFN \#1}
\label{tab:nfn1}
\begin{tabularx}{\textwidth}{lX}
\hline
\rowcolor[HTML]{EFEFEF} 
NFN\#1        & Nichtfunktionale Anforderung \#1\\ \hline
Name          & Daten/State/Verhalten sichtbar machen\\ \hline
Beschreibung  & Für den Endnutzer sollen die Daten, welche das Programm verwenden, der State indem sich das Programm befindet und das Zukünftige verhalten dynamisch sichtbar sein.\\ \hline
Rationale     & Programmcode im IoT ist normalerweise statischer Text, welcher nach einer Inspektion durch den Nutzer kompiliert und vom Computer ausgeführt wird. Der (Nicht-) Erfolg kann dann vom  Nutzer beurteilt werden. Diese Programmierungsprozess ist Gewöhnungsbedürftig, da Menschen visuell denken. Eine Darstellung von Daten, Zustand und Verhalten kann die Ausführung eines Programms für Novizen vereinfachen \\ \hline
Fit-Kriterium & Wenn Daten/State/Verhalten in einer Weise sichtbar sind, in der es für den Endnutzer den Sinn eines Programms in kurzer Zeit erklärlich ist.\\ \hline
\end{tabularx}
\end{table}

\subsubsection{Erlernbarkeit}
\begin{table}[H]
\caption{NFN \#2}
\label{tab:nfn2}
\begin{tabularx}{\textwidth}{lX}
\hline
\rowcolor[HTML]{EFEFEF} 
NFN\#2        & Nichtfunktionale Anforderung \#2 \\ \hline
Name          & Schneller erlernbar als Standard-IoT-Programmiersprachen \\ \hline
Beschreibung  & flowws soll für den fachfremden Endnutzer (bspw. Laura) schneller erlernbar sein, als übliche Programmiersprachen, die im IoT Bereich eingestetzt werden (bspw. C)\\ \hline
Rationale     & Sprachen die im Embedded Bereich verwendet werden haben eine äußerst Steile lernkurve. Oftmals müssen komplexe Paradigmen wie Speicherverwaltung oder Parallelität über Monate trainiert werden bevor man effektiv Arbeiten kann. \\ \hline
Fit-Kriterium & Wenn Laura an einem Tag lernt die Grundzüge von flowws deuten, nach einer Woche ein vorgefertigtes Szenario programmieren und nach einem Monat ihre eigenen Szenarios entwickeln kann, ist die einfach Erlernbarkeit von flowws nachgewiesen. \\ \hline
\end{tabularx}
\end{table}

\subsubsection{Bedienbarkeit}
\begin{table}[H]
\caption{NFN \#3}
\label{tab:nfn3}
\begin{tabularx}{\textwidth}{lX}
\hline
\rowcolor[HTML]{EFEFEF} 
NFN\#3        & Nichtfunktionale Anforderung \#3 \\ \hline
Name          & Schnelle und Flexible Programmierung \\ \hline
Beschreibung  & Die Programmierung von funktionalen Prototypen soll bei kleineren Prototypen leicht von der Hand gehen und unkompliziert modifizierbar sein. \\ \hline
Rationale     & (Funktionale) Prototypen stellen oftmals ein Teilsystems einer größeren Komponente dar. Aus diesem Grund ist das vergleichsweise langwierige Aufsätzen von IoT Software-Projekten zu aufwändig, dies verringert die dynamik innerhalb der Prototypenentwicklung als Ganzes. \\ \hline
Fit-Kriterium & Ein (funktionaler) Prototyp mit fünf oder weniger Sensoren/Aktoren sollte in wenigen Stunden programmierbar sein.   \\ \hline 
\end{tabularx}
\end{table}

\subsubsection{Zuverlässigkeit/Fehlertoleranz}
\begin{table}[H]
\caption{NFN \#4}
\label{tab:nfn4}
\begin{tabularx}{\textwidth}{lX}
\hline
\rowcolor[HTML]{EFEFEF} 
NFN\#4        & Nichtfunktionale Anforderung \#4 \\ \hline
Name          & Prävention von Übersetzungszeit-Fehlern \\ \hline
Beschreibung  & Fehler sollen dadurch vermieden werden, dass bspw. ungeeignete Datentypen nicht miteinander verrechnet werden können. \\ \hline
Rationale     & Viele EUDs erlauben es, ''kreativ'' mit dem verrechnen von Datentypen. In SAM Labs ist es bspw. möglich, integer mit boolean Daten  über logische Gatter miteinander zu verbinden. Ein solches Konstrukt macht logisch keinen Sinn und fordert unvorhergesehenes Verhalten geradzu heraus. \\ \hline
Fit-Kriterium & Übersetzungszeitfehler (bspw. Typisierung) werden präventiv verhindert.   \\ \hline
\end{tabularx}
\end{table}

\subsubsection{Änderbarkeit/Wartbarkeit}
\begin{table}[H]
\caption{NFN \#4}
\label{tab:nfn4}
\begin{tabularx}{\textwidth}{lX}
\hline
\rowcolor[HTML]{EFEFEF} 
NFN\#5        & Nichtfunktionale Anforderung \#5 \\ \hline
Name          & Erweiterung von flowws hinsichtlich Transducern/Sensoren/Aktuatoren \\ \hline
Beschreibung  & Experten und Endnutzer können neue Transducer/Sensoren/Aktuatoren in flowws integrieren und ihre Funktionalitäten für andere Enduser bereitstellen. \\ \hline
Rationale     & EUDs sind vorwiegend geschlossene Systeme. Sie wachsen nur vom Funktionsumfang wenn der Hersteller dies erlaubt. Das führt dazu, dass EUDs schnell an ihre Grenzen stoßen und als Produktivwerkzeug nicht mehr von Nutzen sind. flowws soll dieses Problem mindern, indem es dem Nutzern erlaubt den Funktionsumfang dynamisch zu erweitern. \\ \hline
Fit-Kriterium & Wenn flowws durch Programmcode sich an neue, vorher nicht mögliche, Anforderungen anpassen kann ohne dass der Sourcecode angefasst werden muss, ist diese Anforderung erfüllt. \\ \hline
\end{tabularx}
\end{table}

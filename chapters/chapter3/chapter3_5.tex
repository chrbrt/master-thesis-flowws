\section{Anforderungen}\label{sec:anforderungen}
Dieser Abschnitt handelt über die funktionalen und nicht-funktionalen Anforderungen von flowws. Da es sich bei flowws um eine Konzeptarbeit handelt, wird der Fokus auf nicht-funktionale Anforderungen gelgt, welche Bedingungen an Benutzbarkeit, Transparenz und an die in Kapitel \ref{tab:cognitivedimensions} beschriebenen Cognitive Dimensions stellen. 

\subsection{Nicht-funktionale Anforderung}

\subsubsection{Verständlichkeit}
\begin{table}[H]
\caption{NFA \#0}
\label{tab:NFA0}
\begin{tabularx}{\textwidth}{lX}
\hline
\rowcolor[HTML]{EFEFEF} 
NFA\#0        & Nichtfunktionale Anforderung \#0 \\ \hline
Name          & Paralleles und sequentielles Verhalten begreifbar machen \\ \hline
Beschreibung  & flowws ermöglicht dem Endnutzer, ein Verständnis von parallelen aperiodischen Ereignissen und sequentiellen Vorgängen, zu erlangen. \\ \hline
Rationale     & Parallele Programmierung ist selbst für Experten ein kognitiv anstrengender Prozess. Sie ist allerdings unabdingbar für den asynchronen, aperiodischen Kontext von cBlocks. Gleichzeitig benötigt der Endnutzer ein solides Paradigma um sequentielle Steuerung von bspw. Aktoren zu verstehen. \\ \hline
Fit-Kriterium & Ein Endnutzer muss in wenigen Stunden das parallele und sequentielle Verhalten von fertigen Programmen deuten und modifizieren können.\\ \hline
\end{tabularx}
\end{table}


\begin{table}[H]
\caption{NFA \#1}
\label{tab:NFA1}
\begin{tabularx}{\textwidth}{lX}
\hline
\rowcolor[HTML]{EFEFEF} 
NFA\#1        & Nichtfunktionale Anforderung \#1 \\ \hline
Name          & Daten und State sichtbar machen \\ \hline
Beschreibung  & Für den Endnutzer sollen die Daten, die das Programm verarbeitet und der Zustand indem sich das Programm befindet für den Endnutzer sichtbar sein. \\ \hline
Rationale     & Programmcode ist normalerweise statisch. Diese Darstellung ist gewöhnungsbedürftig, da Menschen visuell denken. Eine Darstellung von Daten und Zustand kann die Ausführung eines Programms für Laien vereinfachen. \\ \hline
Fit-Kriterium & Wenn Daten und Zustand in einer Weise sichtbar sind, dass das Verhalten eines Programms, in kurzer Zeit für den Endnutzer erklärlich ist. \\ \hline
\end{tabularx}
\end{table}

\subsubsection{Erlernbarkeit}
\begin{table}[H]
\caption{NFA \#2}
\label{tab:NFA2}
\begin{tabularx}{\textwidth}{lX}
\hline
\rowcolor[HTML]{EFEFEF} 
NFA\#2        & Nichtfunktionale Anforderung \#2 \\ \hline
Name          & Schneller erlernbar als gebräuchliche Programmiersprachen im \ac{IoT} \\ \hline
Beschreibung  & flowws soll für den fachfremden Endnutzer schneller erlernbar sein, als übliche Programmiersprachen, die im IoT Bereich verwendet werden (bspw. C).\\ \hline
Rationale     & Programmierprachen die im Embedded-Bereich verwendet werden, besitzen eine flache Lernkurve. Komplexe Paradigmen wie Speicherverwaltung oder Parallelität müssen über Monate trainiert werden, bevor man effektiv mit ihnen umgehen kann. \\ \hline
Fit-Kriterium & Wenn Laura an einem Tag lernt die Grundzüge von flowws deuten kann, nach einer Woche ein vorgefertigtes Szenario programmieren kann und nach einem Monat ihre eigenen Szenarios entwickeln kann, gilt die einfache Erlernbarkeit von flowws als erwiesen. \\ \hline
\end{tabularx}
\end{table}

\subsubsection{Bedienbarkeit}
\begin{table}[H]
\caption{NFA \#3}
\label{tab:NFA3}
\begin{tabularx}{\textwidth}{lX}
\hline
\rowcolor[HTML]{EFEFEF} 
NFA\#3        & Nichtfunktionale Anforderung \#3 \\ \hline
Name          & Schnelle Modifizierung bestehender Programme unterstützen \\ \hline
Beschreibung  & Bestehende Programme sollen nicht nur leicht erfassbar für den Endnutzer sein, sondern auch effizient modifizierbar sein. \\ \hline
Rationale     &   Ein schnelles ändern der Programm-Parameter und Strukturen erlaubt es dem Endnutzer in möglichst wenigen Schritten verschiedene Versionen eines Programms auszuprobieren. Dadurch wird dem Endnutzer exploratives Entwickeln ermöglicht.\\ \hline
Fit-Kriterium & Parameter und Strukturen des flowws-Programmes sind zur Laufzeit änderbar. \\ \hline 
\end{tabularx}
\end{table}

\begin{table}[H]
\caption{NFA \#4}
\label{tab:NFA4}
\begin{tabularx}{\textwidth}{lX}
\hline
\rowcolor[HTML]{EFEFEF} 
NFA\#4        & Nichtfunktionale Anforderung \#4 \\ \hline
Name          & Prävention von Übersetzungszeit-Fehlern \\ \hline
Beschreibung  & flowws soll vermeidbare Fehler, wie Konflikte zwischen Datentypen, im Voraus unterbinden. \\ \hline
Rationale     & Um zügig in der Programmierung voran zu kommen, ist es besser Fehler zu vermeiden ist besser als sie beheben zu müssen. \\ \hline
Fit-Kriterium & Übersetzungszeitfehler (bspw. fehlerhafte Typisierung) werden präventiv verhindert.   \\ \hline
\end{tabularx}
\end{table}

\subsubsection{Änderbarkeit/Wartbarkeit}
\begin{table}[H]
\caption{NFA \#5}
\label{tab:NFA5}
\begin{tabularx}{\textwidth}{lX}
\hline
\rowcolor[HTML]{EFEFEF} 
NFA\#5        & Nichtfunktionale Anforderung \#5 \\ \hline
Name          & Erweiterung von flowws hinsichtlich Transducern, Sensoren und Aktuatoren \\ \hline
Beschreibung  & Experten und Endnutzer können neue Transducer/Sensoren/Aktuatoren in flowws integrieren und ihre Funktionalitäten für andere Enduser bereitstellen. \\ \hline
Rationale     & \ac{EUD}-Werkzeuge sind vorwiegend geschlossene Systeme. Sie wachsen nur vom Funktionsumfang wenn der Hersteller dies erlaubt. Das führt dazu, dass sie schnell an ihre Grenzen stoßen und als Produktivwerkzeug nicht mehr von Nutzen sind. \\ \hline
Fit-Kriterium & Die Anforderung gilt als erfüllt, wenn flowws durch Programmcode sich an neue, vorher nicht mögliche, Anforderungen anpassen kann. Dies geschieht, ohne dass der Endnutzern den Sourcecode verändern muss. \\ \hline
\end{tabularx}
\end{table}

\subsection{Funktionale Anforderungen}\label{subsec:fanf}
\begin{table}[h]
\caption{Funktionale Anforderungen}
\label{tab:fanf}
\begin{tabularx}{\textwidth}{llX}
\hline
\rowcolor[HTML]{EFEFEF}
ID    & Name                       & Beschreibung \\ \hline
FA\#1 & Vergleichsoperationen      & Verechnung von reellwertigen Daten durch ($<,\leq,=,\neq,\geq,>$) möglich. \\ \hline
FA\#2 & Bool'sche Operationen      & Verechnung von bool'schen Daten durch ($\neg, \land, \lor, \oplus$) möglich. \\ \hline
FA\#3 & Zeitgestauerte Operationen & Zeitliche Manipulation von Signalen durch Verzögerung \\ \hline
FA\#4 & Auslesen von Sensordaten   & Kontinuierliches Auslesen von Sensordaten zur Weiterverarbeitung \\ \hline
FA\#5 & Ansteuern von Aktoren      & Ansteuern von Aktoren und Auslesen deren Zustände \\ \hline
FA\#6 & Programmierung von Aktoren & Verhalten von Aktoren ist programmierbar \\ \hline
FA\#7 & Konvertierung von Signalen & Konvertierung zwischen verschiedenen Wertebereichen (bspw. $\left [ 0;1,0 \right ] \rightarrow \left [ 10;37 \right ]$) \\ \hline
\end{tabularx}
\end{table}
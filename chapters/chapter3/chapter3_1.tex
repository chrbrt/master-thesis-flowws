\section{Problemanalyse}\label{sec:problemanalyse}
Im vorherigen Kapitel wurden die einzelnen Aspekte aufgearbeitet, aus denen sie der Kontext der \ac{EUD}-Umgebung zusammensetzt. Im Folgenden werden daraus die einzelnen Probleme hergeleitet, mit welcher sich das in dieser Arbeit beschriebene Konzept beschäftigt.

\subsection{Parallel contra Sequentiell -- Datenfluss contra Kontrollfluss}
Wie in den vorherigen Kapitel beschrieben lässt sich die Kommunikation zwischen den einzelnen Nodes eines \ac{IoT}-System als \textit{aperiodisch} und \textit{parallel} beschreiben. Sensoren in einem solchen Verbund senden abhängig ihrer Abtastraten asynchron Informationen als Datenpakete über ein \textit{Shared Medium} zu Signalwandlern (Transducer). Dieses Verhalten, welches die Daten selbst und die Transformation dieser in den Fokus legt beschreibt einen Datenfluss von Erzeuger zu Konsument. \acp{EUD} wie beispielsweise LabVIEW nutzen hierbei  Datenflussdiagramme um den Benutzer ein organisieren und transformieren von Datenströme zu ermöglichen. Dies ist allerdings für abgeschlossene \ac{IoT}-Systeme, wie sie mit cBlocks designt nur zum teil nützlich. Die Datenströme werden nämlich von Aktuatoren konsumiert. Da Aktuatoren natürlich zu jeder Zeit nur einen \textit{State} annehmen können (bspw. kann ein Motor sich nicht parallel in zwei Richtungen sich drehen) ist die Verarbeitung der Eingangsignale \textbf{sequentiell} anhand eines definierten Kontrollflusses. Hier liegt ein zentrales Problem: während die Orchestrierung zwischen den Nodes anhand von parallelen Datenflüssen modelliert wird stellt sich die Signalverarbeitung bei den Aktuatoren als ein sequentieller Kontrollfluss dar. Eine \ac{EUD}-Lösung für cBlocks muss demnach beide Paradigmen effizient abbilden können: das Orchestrieren und Transformieren von parallelen Datenströmen sowie das Modellieren von sequentiellen Verhalten von Aktuatoren.

\subsection{Signalpriorisierung}
Bei der Signalpriorisierung handelt es sich um das Verhalten eines Datenkonsuments, wenn zwei, in Konflikt stehende Nachrichten eintreffen. Dieses Problem kann im folgenden Szenario verdeutlicht werden:
\begin{quote}
''Wenn es 32°C heiß ist, dann lass die LED rot aufleuchten.'' \\
''Wenn es regnet, dann lass die LED blau aufleuchten.''
\end{quote}
Es wird schnell klar, dass beide Regeln in Konflikt zueinander stehen. Sollte es regnen und 32°C heiß sein ist nicht klar, welche Farbe die LED hat. Falls das System keinen klar definierten Umgang mit solchen kollidierenden Signalen besitzt, fällt es in einen nicht vorhersehbaren State. \acp{EUD} besitzen daher unterschiedliche Mechanismen dieses Dilemma zu lösen: bei IFTTT beispielsweise gibt die Reihenfolge der Regeln an, welche Regel angewandt wird, bei SAM Space überschreibt das neueste Signal, das Letzte. Auch andere Systeme wie das explizite Definieren von Priorisierungsregeln ist erdenklich, es muss nur in Anbetracht gezogen werden, dass ohne eine solche Regelung das System intransparent für den Endnutzer wird und eine Fehlersuche erschwert.

\subsection{Visualisierung von Daten}
\begin{quote}
    ''\textit{Visualize Data, not Code. Dynamic behaviour, not static structure}'' \\ -- Bret Victor, Beast of Burden
\end{quote}
In \ac{IoT} sind Daten Dreh- und Wendepunkt für das Design von Systemen: Daten werden an den Sensoren erhoben von Wandlern transformiert und von Aktuatoren konsumiert. Es macht daher Sinn den Fokus der Visualisierung auf die Daten selbst zu setzen. Das Problem hierbei ist eine sinnvolle Darstellung für die Daten und deren Transformation zu finden. Während IFTTT dieser Schwierigkeit durch einen vereinfachten Scope weitestgehend umgeht benutzt SAM Labs animierte graphische Elemente um den Kontrollfluss darzustellen. Beide \ac{EUD}-Systeme stellen nur bedingt die Datenflüsse über eine zeitliche Dimension dar, dies erschwert die Fehlersuche da das Verhalten des Programmartefakts schwerer nachvollziehbar ist. Für professionelle Software-Engenieure ist ein solches Visualisieren der Daten unabdingbar. Das Verwenden von Debugging-Werkzeugen, welche diese Visualisierung ermöglichen, beherrschen den Programmier-Alltag. Für eine \ac{EUD} eröffnet sich daher das Problem, wie dieses Daten visualisiert werden können um das Verhalten und den State des Programmartefakt auch für Laien erklärbar erscheint. 

\subsection{Funktionsumfang contra Komplexität}
Ein Beweggrund für die Entwicklung der cBlocks ist das geringe Maß an Komplexität, welches ähnlichen Produkte wie SAM Labs oder Sonys MESH\footnote{\url{http://meshprj.com/en/}, besucht am 24.06.2018} zulässt. Der Grund hierfür ist, dass diese und viele weitere Produkte ihren Fokus auf die Vermittlung von MINT-Wissen legen und ihre Produkte an vorgefertigte Szenarien binden, welche von den End-Nutzern nachgebildet werden sollen. Im Kontext des freien Prototypings wird (im Selbstversuch) allerdings schnell klar, dass diese Werkzeuge auf Soft- und Hardwareebene an ihre Grenzen stoßen. Wie schon im Grundlagen-Kapitel erläutert, muss beim Design einer \ac{EUD} deren möglichen Umfang an abbildbarer Programmlogik mit der Darstellungskomplexität ausbalanciert werden. In SAM Labs wird die gewählte Balance bei der Steuerung von Aktuatoren verdeutlicht. Hierbei wird das Verhalten von Aktuatoren auf Eingangssignale nicht selbst gesteuert. Vielmehr werden vorgefertigte Steuerungsblöcke, welche für jeden Aktuator speziell vorgefertigt sind. Im Falle eines LED-Aktuators währen dies Funktionen wie ''Farbe rotieren'' oder ''Helligkeit ändern''. Dies erleichtert zwar die Programmierung schränkt, sobald man für den Anwendungsfall speziellere Verhalten benötigt, ein. Das Problem ist also, das bestehende Produkte nicht mit den erlernten Fähigkeiten des Endnutzers wachsen, sondern lediglich ein Einstieg in die Programmierung der \ac{IoT}-Domäne darstellen. Diese Werkzeuge werden (bewusst) obsolet sobald ein Maß von Programmkomplexität erreicht wird. Ein daraus resultierendes Folgeproblem ist, das Werkzeuge wie SAM-Labs nur bedingt den tatsächlichen Programmablauf abstrahieren und somit echte, für die Programmierung von \ac{IoT}-Geräten relevante Konzepte, wie \textit{State-Machines} oder \textit{Event-Loops} nicht vermitteln.

\subsection{Open Source}
Das Problem von geschlossenen/proprietären Systemen ist nicht nur für \acp{EUD} sondern auch für cBlocks ähnliche Projekte, die es erlauben \ac{IoT}-Projekte zu entwickeln, relevant. Bausätze wie SAM-Labs oder Sonys MESH sind weder auf Software- noch Hardwareebene erweiterbar. Dies schränkt den möglichen Funktionsumfang signifikant ein und macht eine Erweiterung und Fortentwicklung der Plattform abhängig vom Hersteller. Währenddessen wurde das cBlocks-Projekt auch darauf ausgelegt, dass Endnutzer die Möglichkeit besitzen, generische Blöcke durch Sensoren auf den Anwendungsfall anpassbar zu machen. Dies verlangt der \ac{EUD}-Umgebung und Hardware eine höheren Erweiterungsgrad ab. Es lässt sich vermuten, dass die Einschränkungen von bestehenden Werkzeugen aufgrund von Geschäftsmodellen existieren. Das Problem, welches hieraus resultiert ist, dass sie als Prototyping-Werkzeuge nur eingeschränkt verwendbar sind, da sich der Endnutzer dem Funktionsumfang des Werkzeugs anpassen muss -- umgekehrt, wie es sein sollte. Dies ist in einem Kontext, in dem eine \ac{EUD}-Umgebung als Vermittlungswerkzeug für generelles MINT-Lehrmaterial dient verständlich, nicht aber, wenn Designer Prototypen in spezialisierten Anwendungsdomänen bauen müssen.

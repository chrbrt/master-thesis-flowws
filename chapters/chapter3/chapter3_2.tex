\section{Persona}\label{sec:persona}
Personas stellen die Verkörperung der Endnutzers dar, welche die cBlocks und somit auch die \ac{EUD} verwenden sollen. Da \textit{flowws} als ein Teilsystem des cBlocks-Produkts entwickelt wird ist es sinnvoll, die in \cite{weckbach2018cblocks} beschriebene Persona aufzugreifen. Im Folgenden werden zwei verschiedene Personas definiert, welche sich in der Komplexität ihrer Ziele, der Größe ihrer Expertisen und ihrem Charakter unterscheiden.

\begin{tcolorbox}[title={Persona \#1, Laura, 24, UX-Designerin},toptitle=3mm,bottomtitle=3mm, bicolor ,sidebyside,righthand width=3cm, sharp corners, boxrule=.4pt, colback=green!5, colbacklower=green!5, label=personalaura]
\begin{quote}
    ''\textit{Ich möchte meine Gedanken so schnell wie möglich greifbar machen.}''
\end{quote}
    \textbf{Ziele:} 
    \setlist[1]{itemsep=-5pt}
    \begin{itemize}
        \item Arbeit produktiv verbringen, anstatt Werkzeuge zu erlernen
        \item Erstellung von \textit{Gadgets} für den privaten Gebrauch
        \item Erprobung der Benutzerfreundlichkeit von \textit{Smart-Device}-Prototypen
    \end{itemize}
    \textbf{Expertise:} 
    \setlist[1]{itemsep=-5pt}
    \begin{itemize}
        \item \ac{UX}-Design Prinzipien, graphische Konzeptualisierung
        \item Design-Thinking, Double Diamond, etc.
        \item Paper-Prototyping, Wizard-of-Oz, etc.
        \item Photoshop, Illustrator, Sketch, etc.
        \item Basiswissen in Arduino/C, Javascript
    \end{itemize}
    
    \tcblower
    
    \includegraphics[width=\linewidth]{bilder/chapter3/laura.png}
\end{tcolorbox}
Die zwei maßgeblichen Stakeholder des Projekts sind Laura (Persona \#1) und Mark (Persona \#2). Ihre Gemeinsamkeit ist, dass sie in der selben Consulting-Firma angestellt sind die sich darauf spezialisiert hat, für Drittfirmen Case-Studies im Bereich Produktentwicklung zu erstellen. Der Berufsalltag beider besteht darin, moderne Design-Methoden wie Design-Thinking auf Probleme von Industriefirmen anzuwenden. Die Arbeit besteht hierbei aus kurzen Entwicklungszyklen, in denen mehrere Prototypen gebaut werden. Hierbei arbeiten beide Personas Hand-in-Hand um über mehrere Iterationszyklen hinweg, um einen finalen Produktprototypen zu erstellen.

Der Unterschied beider Personas sind ihre Pain-Points, Ziele und Expertiesen. Laura lässt ihre Erfahrung in User Experience in die Entwicklung von Produkt-Hardware einfließen. Markus Hintergrund ist technischer Natur. Er lässt seine Erfahrung in Elektrotechnik und Embedded-Entwicklung auf das Produkt Design einfließen. Beide Personas arbeiten somit im Tandem, um \ac{UX} und Hardware miteinander zu verschmelzen. Lauras größter Pain-Point hierbei ist, dass ein iteratives Entwickeln von funktionalen Prototypen, durch die enorme technische Komplexität von Embedded-Hardware gehemmt wird. Obwohl sie grundsätzliche Erfahrung mit Arduino besitzt, können selbst geringfügige Änderungen zu kryptischen Fehlern und unvorhergesehenen Verhaltensweisen des Prototypen führen. Hierbei assistiert Mark, ein Creative Technologist, der sich um 

\begin{tcolorbox}[title={Persona \#2, Mark, 28, Creative Technologist},toptitle=3mm,bottomtitle=3mm, bicolor ,sidebyside,righthand width=3cm, sharp corners, boxrule=.4pt, colback=green!5, colbacklower=green!5]
    \begin{quote}
        ''\textit{Mehrmals den selben Code schreiben ist leider Alltag}''
    \end{quote}
    \textbf{Ziele:} 
    \setlist[1]{itemsep=-5pt}
    \begin{itemize}
        \item Besseres Verwenden von bestehendem Code
        \item Einfacheres Vermitteln von erzeugtem Code mit Designern
        \item Schnelles Iterieren von Prototyp-Generationen
    \end{itemize}
    \textbf{Expertise:} 
    \setlist[1]{itemsep=-5pt}
    \begin{itemize}
        \item Design-Thinking, Rapid-Prototyping
        \item Embedded Programmierung, Elektrotechnische Grundlagen, Software Engineering
        \item Visual Code, Eagle, UML, C, etc.
    \end{itemize}
    \tcblower
    \includegraphics[width=\linewidth]{bilder/chapter3/mark.png}
\end{tcolorbox}

die technische Implementierung der Anforderungen kümmert. Seine Arbeit besteht darin, geeignete elektronische Komponenten zusammenszutellen, die benötigte Software aufzusetzen, Leiterplatten zu entwerfen und komplexe Programmlogik zu schreiben. Für ihn ist es frustrierend, wenn er viel Zeit in immer wiederkehrende Aufgaben, wie das Aufsetzen von generischen Hardwareblöcken, investieren muss. Alleine das Aufsetzen von kleinen funktionalen Prototypen mit nur wenigen Teilen kann durch Design, Erstellung und Fehlerkorrektur von Programmcode, mehr als einen Tag in Anspruch nehmen. Mark würde sich viel lieber mit der Entwicklung von komplexeren Systemen beschäftigen. Laura auf der anderen Seite, wäre es lieber, unabhängiger von Mark zu sein und sich auf die Erprobung funktionaler Prototypen fokussieren zu können, ohne viel Zeit für Elektronik und Software-Debugging zu verschenken. Kommunikation bzw. der Austausch von fachlichen Informationen zwischen den beiden Personas spielt hierbei eine Rolle.

Das Ziel von ''Profi''-Projekt, cBlocks und in Folge dessen, flowws ist es, die Pain-Points der Personas zu mildern, indem sie auf ihre fachlichen Stärken bauen. 
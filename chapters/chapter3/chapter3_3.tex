\section{Vision und Ziel}
Im folgenden werden die Vision für das zu Konzipierende \ac{EUD}-Werkzeug entwickelt welches sich an den Problemen von bestehenden Systemen (ka) und an den Bedürfnissen der Stakeholder (kap) orientieren. Die Vision für flowws beschreibt die gewählte mittel- bis langfristige Zukunft für ein Projekt und erlaubt es Designentscheidungen zu begründen und zu lenken. Die Vision für flowws lautet wie folgt:

\textit{flowws ist ein Werkzeug, das Designern ermöglicht sich in \ac{IoT} auszudrücken wie es ein Stift ihnen ermöglicht sich auf Papier auszudrücken; dadurch schnell, agil und ohne im Sumpf der Technologien zu versinken, funktionale Prototypen zu erstellen.}

Im folgenden werden Ziele des Projekts definiert, welche wenn erreicht, diese Vision umsetzen sollen. 

\begin{table}[H]
\caption{Ziel \#1}
\label{tab:ziel1}
\begin{tabularx}{\textwidth}{lX}
\hline
\rowcolor[HTML]{EFEFEF} 
Ziel \#1:     & Z\#1   \\ \hline
Name          & Schnell zu erlernen \\ \hline
Rationale     & IoT-Entwicklung benötigt ein immenses vertikales Wissen. Man benötigt Erfahrung in Hardware-naher Entwicklung, Programmiersprachen wie C,Basiskenntnisse in Elektrotechnik sowie Mikroelektronik nur um grundlegende Experimentedurchzuführen. Während cBlocks dass erlernen der elektrotechnischen Fähigkeiten erleichtertsoll flowws die Barriere für Programmierung in IoT-Szenarien erleichtern. \\ \hline
Fit-Kriterium & Laura sollte es möglich sein in weniger als ein bis zwei Tagen die Grundlagen von flowws zu verstehen und schon nach wenigen Stunden das grundsätzliche Verhalten von bestehenden Projekten deuten können. \\ \hline
\end{tabularx}
\end{table}

\begin{table}[H]
\caption{Ziel \#2}
\label{tab:ziel2}
\begin{tabularx}{\textwidth}{lX}
\hline
\rowcolor[HTML]{EFEFEF} 
Ziel \#2:     & Z\#2 \\ \hline
Name          & Schnell zu programmieren/modifizieren\\ \hline
Rationale     & Programmierung in IoT-Szenarien artet aufgrund seiner parallelen Event-basierten Programmierstiels und seiner Abhängigkeit von physikalischen Ereignissen nicht selten in Trial-and-Error Vorgehen aus. Selbst minimale Modifikationen können zu unvorhergesehenen Verhalten führen. Aus diesem Grund ist schriftliche Programmierung in Situation bei denen mehrere Parteien in Kooperation arbeiten (bspw. Modifizierung von Code während eines Meeting) hinderlich. \\ \hline
Fit-Kriterium & flowws sollte es dem Nutzer ermöglichen ähnlich Agil zu Programmieren wie der Design-Prozess selbst ist. Aus diesem Grund sollte flowws expermentelles Vorgehen unterstützen und dem Nutzer es ermöglichen (abhängig der Größe des Prototypen)  in ein bis zwei Tagen einen vorläufigen Prototyp zu erstellen. \\ \hline
\end{tabularx}
\end{table}

\begin{table}[H]
\caption{Ziel \#3}
\label{tab:ziel3}
\begin{tabularx}{\textwidth}{lX}
\hline
\rowcolor[HTML]{EFEFEF} 
Ziel \#1:     & Z\#3 \\ \hline
Name          & Expressiv \\ \hline
Rationale     & Es soll dem Nutzer schnell verständlich sein was ein Programm macht und wie es Das macht. Programmcode in seiner rohen Form ist Eindimensional und in seiner Aussagekraft stark Abhängig von der Domäne und der Sorgfallt des Erstellers. In IoT Projekten, wie denen von Laura und Mark arbeiten viele Personen kooperativ arbeiten. Hierbei ist es von Nöten dass ein Programm so gut wie möglich selbst erklären kann. \\ \hline
Fit-Kriterium & Laura und Mark sollen es beide möglich sein ein ihnen unbekanntes Programm innerhalb weniger Minuten (abh. von der Größe des Projekts) zu analysieren und eine zumindest wage Aussage treffen können über das Verhalten des Programms. \\ \hline
\end{tabularx}
\end{table}

\begin{table}[h]
\caption{Ziel \#4}
\label{tab:ziel4}
\begin{tabularx}{\textwidth}{lX}
\hline
\rowcolor[HTML]{EFEFEF} 
Ziel \#4:     & Z\#4 \\ \hline
Name          & Wächst mit dem Nutzer \\ \hline
Rationale     & Viele EUDs sind auf eine eingeschränkte Domäne ausgelegt und daher, von ihrem Leistungsumfang stark begrenzt. Diese Einschränkungen sind für kreative Prozesse suboptimal und stoßen schnell an ihre funktionale Grenze. flowws soll durch Erweiterbarkeit und einen gut gewählten Grad an Allgemeingülltigkeit mit den Fähigkeiten mit den Fähigkeiten des Nutzer wachsen. \\ \hline
Fit-Kriterium & Das Ziel ist erreicht, wenn eine Reihe von unterschiedlichen funktional unterschiedlichen Prototypen mit ein und dem selben Werkzeug (flowws) gebaut werden können. Des Weiteren soll Mark es möglich sein flowws um wiederverwendbare Funktionen und Elemente erweitern abh. von Lisas Spezifikationen erweitern kann.  \\ \hline
\end{tabularx}
\end{table}

Ziel \#1-\#4 sind eine Dekomposition der Vision in einzelne verifizierbare Pakete. Es ist allerdings unrealisitsch, dass all diese Ziele innerhalb dieser Thesis abgearbeitet bzw. verifiziert werden können. Nichts desto trotz helfen die Ziele Anforderungen an das System zu stellen und Grundsätzliche Designentscheidungen zu begründen.


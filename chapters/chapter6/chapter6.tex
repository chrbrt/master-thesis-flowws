\chapter{Reflexion \& Zukünftige Arbeiten}
Im letzten Kapitel wird noch einmal über die Inhalte und Ergebnisse dieser Arbeit reflektiert. Zuletzt wird das Erreichen der Forschungsfragen und die daraus resultierenden Folgefragen diskutiert.

Diese Arbeit hatte zum Ziel, das Konzept für ein \ac{EUD}-Werkzeug zu erstellen, dass die spezifischen Anforderungen von \ac{IoT} und den Projekt-Kontext der \acp{cBlock} aufgreift. Die Arbeitet hat sich aus diesem Grund in vier Teile aufgegliedert:

\begin{itemize}
    \item \textbf{Wissenserhebung} In Kapitel \ref{GrundlagenSOTA} wurden die fachlichen Grundlagen in \ac{IoT} und \ac{EUD} gelegt. Hierbei wurden die unterschiedlichen Aspekte, Komponenten und Charakteristiken von \ac{IoT} definiert. Ziel hierbei war es, ein Gefühl für \textit{Smart Devices} und ihre Art der Kommunikation zu erlangen, um später eine adäquate Metapher für flowws, definieren zu können. Ebenfalls wurde der Begriff der \ac{EUD} geklärt. Das Ergebnis des Kapitels, korrespondiert direkt mit der ersten Forschungsfrage und kulminiert mit einer Analyse von drei populären \ac{EUD}-Paradigmen und ihre Implementierungen in der \ac{IoT}-Domäne.
    \item \textbf{Analyse} Kapitel \ref{chapter:analyse} bestand aus einer zweigeteilten Analyse, die die generellen Probleme von \acp{EUD} im \ac{IoT} zu abstrahieren und den Endnutzer selbst mit seinen Problemen zu charakterisieren. Das Ergebnis zielt auf die Beantwortung der Forschungsfrage xxx ab und ist eine Spezifikation von Vision, Zielen und Anforderungen eines \ac{EUD}-Werkzeugs, welches sich Gesamtkontext dieser Thesis orientiert.
    \item \textbf{Konzeption} Das Ziel von Kapitel \ref{chapter:konzept} war es, sämtliche Probleme und Anforderungen in ein Programmierkonzept umzusetzen. Das Ergebnis wurde ''flowws'' getauft. Es ist eine graphische Programmiersprache Datenfluss-Logik und \ac{FSM}-Logik zu kombiniert um somit die spezifischen Rahmenbedingungen des Kontext erfüllt und somit Forschungsfrage (XXX) beantwortet.
    \item \textbf{Evaluation} Die Evaluation des Konzeptmodells wurde anhand eines Nutzertests mit einem \ac{EUD}-Prototyp und geeigneten Endnutzern, durchgeführt. Ziel hierbei war es festzustellen, ob das Konzeptmodell von flowws mit dem mentalen Modell der Endnutzer vereinbar ist. Das Ergebnis ist die Antwort auf Forschungsfrage (XXX), indem bewiesen wurde, dass flowws eine vielversprechendes Konzeptmodell besitzt, welches die fachlichen und technischen Anforderungen unterstützt und gleichzeitig den Endnutzer in kurzer Zeit nachvollziehbar ist.
\end{itemize}

Das Endprodukt dieser Thesis ist das \ac{EUD}-Konzept, das hinter flowws steht. Das Konzept hat gezeigt, dass eine Kombination von Datenfluss- und \ac{FSM}-Pro\-gram\-ma\-tik nicht nur eine treffende Abstraktion bzw. Metapher für \ac{IoT} ist, sondern auch ein guter Kompromiss zwischen Funktionsumfang und Erlernbarkeit bietet. Es wurde ein Prototyp entwickelt, welches die Nachvollziehbarkeit des Konzepts bewiesen hat. Dies ist ein Ansatzpunkt für weitere Arbeiten in diesem Kontext:

\begin{itemize}
    \item \textbf{Funktionelle Konzeption:} Eine komplettes \ac{EUD}-Werkzeug ähnelt einer \ac{IDE}. Sie besitzt mehr Funktionalitäten, die der Endnutzer ausnutzen kann, um Programme zu erstellen, zu testen und zu verwalten. Daraus folgend, 
    
    \item \textbf{Technische Konzeption:} Die momentane Implementierung von flowws beschränkt sich auf einen \textit{Experience-Prototyp}. Dieser ist eine reine Darstellung des Konzepts und dient um die \ac{UX} von flowws zu evaluieren. Die technische Umsetzung von flowws wurde in dieser Thesis nicht behandelt, ist aber aufgrund der Interaktionen verschiedener Teilsysteme ein nicht-triviale Problemstellung. Eine technische Anforderungsanalyse, Architektur und Implementierung von flowws kann Teil zukünftiger Arbeiten sein. 
    
    \item \textbf{Evaluation} größere evaluation - workshop - evaluation zur erstellung statt rein zum anschauen
\end{itemize}



short comings/not behandelte themen -> Fragen für nächste arbeiten
\chapter{Reflexion \& Zukünftige Arbeiten}
Im letzten Kapitel wird noch einmal über die Inhalte und Ergebnisse dieser Arbeit reflektiert. Zuletzt wird das Erreichen der Forschungsfragen und die daraus resultierenden Folgefragen diskutiert.

Diese Arbeit hatte zum Ziel, das Konzept für ein \ac{EUD}-Werkzeug zu erstellen, dass die spezifischen Anforderungen von \ac{IoT} und den Projekt-Kontext der \acp{cBlock} aufgreift. Die Arbeitet hat sich aus diesem Grund in vier Teile aufgegliedert:

\begin{itemize}
    \item \textbf{Wissenserhebung:} In Kapitel \ref{GrundlagenSOTA} wurden die fachlichen Grundlagen in \ac{IoT} und \ac{EUD} gelegt. Hierbei wurden die unterschiedlichen Aspekte, Komponenten und Charakteristiken von \ac{IoT} definiert. Ziel hierbei war es, ein Gefühl für \textit{Smart Devices} und ihre Art der Kommunikation zu erlangen, um später eine adäquate Metapher für flowws, definieren zu können. Ebenfalls wurde der Begriff der \ac{EUD} geklärt. Das Ergebnis des Kapitels, ist eine Analyse von drei populären \ac{EUD}-Paradigmen und ihre Implementierungen in der \ac{IoT}-Domäne. Es wurde herausgefunden, dass bestehende \ac{EUD}-Systeme in ihrem Design stark von Anwendungsfall (bspw. Lehre) oder ihrer Art von Endnutzer (bspw. Privatpersonen) abhängig sind. Aus diesem Grund wurden bestehende Lösungen für die Erstellung von \ac{IoT}-Prototypen durch Designer, für unpassend befunden. 
    \item \textbf{Analyse:} Kapitel \ref{chapter:analyse} bestand aus einer zweigeteilten Analyse, die die generellen Probleme von \acp{EUD} im \ac{IoT} abstrahiert und dem Endnutzer selbst mit seinen Problemen charakterisiert. Das Ergebnis zielt auf die Beantwortung der Forschungsfrage \#1 (siehe Kapitel \ref{sec:1_zielsetzung}) ab und ist eine Spezifikation von Vision, Zielen und Anforderungen eines \ac{EUD}-Werkzeugs, welches sich Gesamtkontext dieser Thesis orientiert. Es wurden Probleme identifiziert, die ein \ac{EUD}-Werkzeug in der Domäne des \ac{IoT}-Prototyping zu bewältigen. Hierzu zählen u.a. die Modellierung von intuitive Programmierung von sequentieller und paralleler Logik oder die Erweiterbarkeit des Systems durch den Endnutzer. Aus diesen Problemen entstand die Vision von flowws, ein \ac{EUD}-Konzept. Es legt Wert darauf, die richtige Balance zwischen Komplexität und Intuitivität zu finden, sodass Designer nicht überfordert aber auch gleichzeitig nicht, von einem geringen Funktionsumfang, kreativ eingeengt werden.
    \item \textbf{Konzeption:} Das Ziel von Kapitel \ref{chapter:konzeption} war es, sämtliche Probleme und Anforderungen in ein Konzeptmodell umzusetzen. Das Ergebnis wurde "`flowws"' getauft. Es ist eine graphische Programmiersprache, die Datenflussparadigma und \ac{FSM}-Logik kombiniert, um die spezifischen Rahmenbedingungen des Kontext zu erfüllen und somit Forschungsfrage \#2 (siehe Kapitel \ref{sec:1_zielsetzung}) beantwortet.
    \item \textbf{Evaluation:} Die Evaluation des Konzeptmodells wurde anhand eines Nutzertests mit einem \ac{EUD}-Prototyp und geeigneten Endnutzern, durchgeführt. Ziel hierbei war es festzustellen, ob das Konzeptmodell von flowws mit dem mentalen Modell der Endnutzer vereinbar ist. Das Ergebnis ist die Antwort auf Forschungsfrage \#3 (siehe Kapitel \ref{sec:1_zielsetzung}), indem gezeigt wurde, dass flowws eine vielversprechendes Konzeptmodell besitzt, welches die fachlichen und technischen Anforderungen unterstützt und gleichzeitig für den Endnutzer in kurzer Zeit nachvollziehbar ist. Es hat auch gezeigt, dass das Verwenden von graphischen Metaphern und geeigneten Animationen komplexe Paradigmen wie parallele Programmierung und \ac{FSM}-Logik, schnell für den Nutzer begreifbar machen.
\end{itemize}

Das Endprodukt dieser Thesis ist das \ac{EUD}-Konzept, das hinter flowws steht. Das Konzept hat gezeigt, dass eine Kombination von Datenfluss- und \ac{FSM}-Pro\-gram\-ma\-tik nicht nur eine treffende Abstraktion für \ac{IoT} ist, sondern auch ein guter Kompromiss zwischen Funktionsumfang und Erlernbarkeit bietet. Durch einen Nutzertest mit einem Prototypen, konnte eine erste, positive Aussage über die Effektivität des Konzepts getroffen werden. Dies ist ein guter Ansatzpunkt für weitere Arbeiten in diesem Kontext:

\begin{itemize}
    \item \textbf{Funktionelle Konzeption:} Eine komplettes \ac{EUD}-Werkzeug ähnelt einer \ac{IDE}. Sie besitzt Funktionalitäten, die der Endnutzer ausnutzen kann, um Programme zu erstellen, zu testen und zu verwalten. Es stellt sich also zur Frage, welche weiteren Funktionen benötigt werden, um aus flowws ein produktives Werkzeug zu machen. Anhand der Ergebnisse des Nutzertests könnten folgende funktionelle Erweiterungen sinnvoll sein:
    \begin{itemize}
        \item \textbf{Simulationsmodus:} Ein Kritikpunkt der Probanden war es, dass Animationen teilweise zu schnell sind. Dies ist in größeren Projekten mit mehreren simultan arbeiteten \acp{cBlock} unvermeidlich. Ein Simulationsmodus, der dem Endnutzer eine Schritt-für-Schritt Ausführung des Graphen ermöglicht, kann hierbei eine Lösung darstellen.
        \item \textbf{Datenvisualisierung:} Ein weiterer Kritikpunkt war, dass die Aktualisierungen der Daten an den Schnittstellen zwar ersichtlich aber nicht historisch nachvollziehbar sind. Eine Möglichkeit dieses Problem zu lösen wäre es, Knoten einzuführen, welche den Verlauf der Daten graphisch Aufzeichnen, bspw. durch Plotten eines Diagramms.
    \end{itemize} 
    
    \item \textbf{Technische Konzeption:} Die momentane Implementierung von flowws beschränkt sich auf einen \textit{Experience-Prototyp}. Dieser ist eine reine Darstellung des Konzepts und dient um die \ac{UX} von flowws zu evaluieren. Die technische Umsetzung von flowws wurde in dieser Thesis nicht behandelt, ist aber aufgrund der Interaktionen verschiedener Teilsysteme eine nicht-triviale Problemstellung. Folglich kann eine technische Anforderungsanalyse, Architekturbeschreibung und Implementierung, die cBlocks integriert, ein Ziel zukünftiger Arbeiten sein. 
    
    \item \textbf{Erweiterte Evaluation:} Eine der möglichen Kritiken der Evaluation ist die Größe der Stichprobe. Drei Partizipanten sind nicht ausreichend, um eine nachhaltige Aussage zu treffen. Nachfolgende Arbeiten könnten die Anzahl der Probanden erhöhen um eine allgemeingültige Aussage über das Konzeptmodell von flowws zu treffen. Des Weiteren war die hier aufgeführte Evaluation rein passiv, sprich der Nutzer hat nicht aktiv mit flowws interagiert. Zukünftige Arbeiten könnten ausführlichere Studien beinhalten, die den Erstellungsprozess von Graphen in flowws auf seine Nutzerfreundlichkeit überprüfen.
\end{itemize}
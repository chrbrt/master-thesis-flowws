\section{Hypothese: Mentales Modell}\label{sec:hypothese}
\begin{quote}
    \textit{flowws besitzt ein intuitives Konzeptmodell. State-Machines und Flow-basierte Programmierung unterstützen das mentale Modell der Stakeholder.}
\end{quote}

In dieser These wird aufgestellt, wie intuitiv die Personas mit flowws umgehen können. Es ist zu erwarten, dass Probanden, welche die Stakeholder repräsentieren nach einer minimalen Einweisung, Programme die in flowws modelliert wurden deuten und das Verhalten von ihnen antizipieren können. Für das Konzept von flowws wurde sich für eine graphische Oberfläche entschieden. Es wurde dadurch erhofft, an die graphische Denkweise vieler Endnutzer anzuknüpfen zu können und dadurch einen größeren Lerneffekt und mehr Verständnis für die Programmlogik im Vergleich zu textuellen Programmcode zu erzeugen. Die \ac{GUI} erweißt sich als effektives Kommunikationsmedium um das Konzeptmodell von flowws mit dem mentalen Modell des Stakeholders zu verbinden.
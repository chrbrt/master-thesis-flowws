% -------------------------------------------------------
% Daten für die Arbeit
% Wenn hier alles korrekt eingetragen wurde, wird das Titelblatt
% automatisch generiert. D.h. die Datei titelblatt.tex muss nicht mehr
% angepasst werden.

\newcommand{\hsmasprache}{de} % de oder en für Deutsch oder Englisch
                              % Für korrekt sortierte Literatureinträge, noch preambel.tex anpassen

% Titel der Arbeit auf Deutsch
\newcommand{\hsmatitelde}{flowws: eine End-User Development Schnittstelle zur Prototypisierung von IoT Systemen}

% Titel der Arbeit auf Englisch
\newcommand{\hsmatitelen}{flowws: an End-User Development Interface for prototyping IoT-Solutions}

% Weitere Informationen zur Arbeit
\newcommand{\hsmaort}{Mannheim}    % Ort
\newcommand{\hsmaautorvname}{Christoph} % Vorname(n)
\newcommand{\hsmaautornname}{Brutscher} % Nachname(n)
\newcommand{\hsmadatum}{xx.xx.2018} % Datum der Abgabe
\newcommand{\hsmajahr}{2018} % Jahr der Abgabe
\newcommand{\hsmafirma}{CAS Software GmbH, Karlsruhe} % Firma bei der die Arbeit durchgeführt wurde
\newcommand{\hsmabetreuer}{Prof. Kirstin Kohler, Hochschule Mannheim} % Betreuer an der Hochschule
\newcommand{\hsmazweitkorrektor}{Prof. Peter Kaiser, Hochschule Mannheim} % Betreuer im Unternehmen oder Zweitkorrektor
\newcommand{\hsmafakultaet}{I} % I für Informatik
\newcommand{\hsmastudiengang}{IM} % IB IMB UIB IM MTB

% Zustimmung zur Veröffentlichung
\setboolean{hsmapublizieren}{true}   % Einer Veröffentlichung wird zugestimmt
\setboolean{hsmasperrvermerk}{false} % Die Arbeit hat keinen Sperrvermerk

% -------------------------------------------------------
% Abstract

% Kurze (maximal halbseitige) Beschreibung, worum es in der Arbeit geht auf Deutsch
\newcommand{\hsmaabstractde}{TODO}

% Kurze (maximal halbseitige) Beschreibung, worum es in der Arbeit geht auf Englisch

\newcommand{\hsmaabstracten}{TODO}

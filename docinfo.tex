% -------------------------------------------------------
% Daten für die Arbeit
% Wenn hier alles korrekt eingetragen wurde, wird das Titelblatt
% automatisch generiert. D.h. die Datei titelblatt.tex muss nicht mehr
% angepasst werden.

\newcommand{\hsmasprache}{de} % de oder en für Deutsch oder Englisch
                              % Für korrekt sortierte Literatureinträge, noch preambel.tex anpassen

% Titel der Arbeit auf Deutsch
\newcommand{\hsmatitelde}{flowws: eine End-User Development Schnittstelle zur Prototypisierung von IoT Systemen}

% Titel der Arbeit auf Englisch
\newcommand{\hsmatitelen}{flowws: an End-User Development Interface for prototyping IoT-Solutions}

% Weitere Informationen zur Arbeit
\newcommand{\hsmaort}{Mannheim}    % Ort
\newcommand{\hsmaautorvname}{Christoph} % Vorname(n)
\newcommand{\hsmaautornname}{Brutscher} % Nachname(n)
\newcommand{\hsmadatum}{xx.xx.2018} % Datum der Abgabe
\newcommand{\hsmajahr}{2018} % Jahr der Abgabe
\newcommand{\hsmafirma}{CAS Software GmbH, Karlsruhe} % Firma bei der die Arbeit durchgeführt wurde
\newcommand{\hsmabetreuer}{Prof. Kirstin Kohler, Hochschule Mannheim} % Betreuer an der Hochschule
\newcommand{\hsmazweitkorrektor}{Prof. Peter Kaiser, Hochschule Mannheim} % Betreuer im Unternehmen oder Zweitkorrektor
\newcommand{\hsmafakultaet}{I} % I für Informatik
\newcommand{\hsmastudiengang}{IM} % IB IMB UIB IM MTB

% Zustimmung zur Veröffentlichung
\setboolean{hsmapublizieren}{true}   % Einer Veröffentlichung wird zugestimmt
\setboolean{hsmasperrvermerk}{false} % Die Arbeit hat keinen Sperrvermerk

% -------------------------------------------------------
% Abstract

% Kurze (maximal halbseitige) Beschreibung, worum es in der Arbeit geht auf Deutsch
\newcommand{\hsmaabstractde}{
Diese Thesis handelt über die Konzeption eines \acl{EUD}-Werkzeuges, das Designer dabei unterstützen soll, schnell, flexibel und mit geringem Lernaufwand die Programmlogik von Prototypen im Bereich der \acl{IoT} zu entwickeln. 

Das Projekt cBlocks (''\textit{connected Blocks}'', siehe \cite{weckbach2018cblocks}) hat zum Ziel, die Erstellung von funktionalen Prototypen im Bereich \acs{IoT} zu erleichtern, indem es die elektrotechnische Komplexität kapselt und für den Endnutzer abstrahiert. Um \textit{Smart Objects}-Prototypen zu erstellen benötigt es allerdings noch eine weitere Komponente: Programmlogik. Das in dieser Thesis erarbeitete \acs{EUD}-Konzept (genannt ''\textit{flowws}''), nimmt sich diesem Problem an. 

Im Verlauf dieser Thesis werden die speziellen Bedürfnisse des Projektkontexts und der involvierten Stakeholder erhoben und analysiert. Auf Basis dieser Erhebung wird eine einzigartiges \acs{EUD}-Werkzeug konzeptioniert, welches auf einer Fusion von Datenfluss- und \acl{FSM}-Logik basiert. Es erlaubt Designern, graphisch Programmlogik zu erstellen und somit die verteilten cBlocks in \textit{Smart Objects} zusammenzuführen. Erste Nutzertests attestieren das Potential von flowws, eine gute Balance zwischen Ausdruckskomplexität und Erlernbarkeit gefunden zu haben.
}

% Kurze (maximal halbseitige) Beschreibung, worum es in der Arbeit geht auf Englisch

\newcommand{\hsmaabstracten}{
The goal of this thesis is the conception of an \acl{EUD}-Tool. This tool aids designers in the creation of \acl{IoT} prototypes by allowing them to program logic quickly ,flexibly and with little knowledge in coding.

The cBlocks-project (''\textit{connected Blocks}'', see \cite{weckbach2018cblocks}) defines a tool with the goal in mind to reduce the required electrotechnical knowledge in order to create \acs{IoT}-prototypes. It does so by encapsulating the technical complexities and thereby abstracting in a way that the enduser (i.e. the designer) can use them effortlessly. However, there is more to a smart device than electronics -- there is logic. Eleviating this problem is the goal the \ac{EUD}-Concept described this thesis.

In the course of this paper the needs of the project's (technical and functional) context and the needs of the stakeholder will be elaborated and analysed. This analysis will culminate in a \acs{EUD}-Tool which is resting on a unique fusion of Dataflow and \acl{FSM} based programming logic. This tool allows designers to graphically create programs which bind the distributed cBlocks together into \textit{Smart Objects}. A user test reveals flowws' potential to balance the expressiveness and learnability to fit the endusers' needs.

}
